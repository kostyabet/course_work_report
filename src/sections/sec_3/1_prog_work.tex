\subsection{Работа программного средства}

К основному функционалу программного средства можно отнести следующие компоненты:
\begin{itemize}
    \item регистрация нового пользователя;
    \item вход в аккаунт;
    \item создание новой задачи;
\end{itemize}

\subsubsection{Регистрация нового пользователя}

Регистрация нового пользователя начинается с получения переданых в запросе параметров и проверка их на корректность.
После получения необходимых параметров, функция проверяет логин пользователя на уникальность: нет ли уже такого логина в системе.
Если данный логин свободен, то необходимо получить логотип пользователя и зашифровать пароль.
После этого создаётся новый пользователь. После этого формируется запись в журнал состояний, что пользователь успешно создан и
возвращается код успшного завершения запроса пользователю.

Если в процессе работы алгоритма произошла непредвиденная ошибка, то будет создана записать в журнал об ошибке и на клиент вернётся
код плохого завершения запроса и краткое описание проблемы.

\subsubsection{Вход в аккаунт}

Если у пользователя уже есть зарегистрированный в системе аккаунт, то ему необходимо в него войти. При запуске приложения у пользователя
появляется страница авторизации в систему, где необходимо ввести свой логин и пароль.

После того как пользователь ввёл логин и пароль, формируется запрос к микросервису авторизации на проверку. Если логин и пароль совпадают
с сущесвующими в системе, то пользователя пускаю в систему под его данными.

\subsubsection{Создание новой задачи}

Для создания задачи авторизованный пользователь должен заполнить, как минимум 3 поля:
\begin{itemize}
    \item название задачи;
    \item дата начала задачи;
    \item дата окончания задачи;
\end{itemize}

Сущесвуют необязательные поля, которые будут описаны в разделе моделей.
Создатель задачи автоматически добавляется, как её участник.

После отправки формы на сервис задач данные проверяются на корректность и в случае возникновения ошибки генерируется
исключение, формируется запись об ошибке в журнал сообщений и возвращается код ошибки пользователю с кратким описанием ошибки. 