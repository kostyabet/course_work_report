\subsection{Модели данных}

Основными моделями в программном средстве {\taskNameFull} являются:
\begin{itemize}
    \item модель пользователя;
    \item модель задачи;
\end{itemize}

Для описания моделей использовалась библиотека Sequalize \cite{sequalizeDocs}, которая написана для удобства работы с базой данных PostgresSQL.

\subsubsection{Модель пользователя}

Модель пользователя используется в сервисе аутентификации и содержит следующие поля:
\begin{itemize}
    \item логин
    \item пароль
    \item имя
    \item фамилия
    \item почта, номер телефона, роль
\end{itemize}

Поля почты, номера телефона и роли используются, как вспомогательные. Использую почту или номер телефона пользователь быстро может
связаться с коллегой. Роль может дать понять пользователям, чем занимается сотрудник.

\subsubsection{Модель задачи}

Модель задачи используется в сервисе задач и содержит в себе ассоциации на другие модели.
Описание модели задачи:
\begin{itemize}
    \item название и описание
    \item приоритет (ассоциация)
    \item статус (ассоциация)
    \item дата начали и дата окончания
    \item участники и прикреплённые файлы
\end{itemize}

Ассоциации необходимы для удобного и гибкого масштабирования моделей. Приоритеты и статусы связываются с моделью задачи и в случае, если
пользователь захочет изменить базовый набор этих моделей, то они автоматически будут применены к модели задачи. Ассоциация производится, по
уникальному идентификатору, который есть и во вспомогательных моделях и в полях основной модели.