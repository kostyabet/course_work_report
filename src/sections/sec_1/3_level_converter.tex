\subsection{Выбор преобразователя уровней} 
Для управления моторами и лазером выбран микроконтроллер ATmega328P в корпусе Arduino Nano.
Данный микроконтроллер использует 5 В логику\cite{schemt_1}; лазер использует питание - 4.7 В, 250 мА; 
а моторы и драйвера двигателей питаются от 12 В. Для того чтобы решить вопрос логических уровней и энергопотребления,
нужно обратиться к теории преобразователей уровней.

Преобразователи уровней\cite{schemt_2} представляют собой специальные компоненты в цифровых устройствах и предназначаются для согласования 
сигналов на входе и выходе по току и напряжению, при применении в одном устройстве интегральных микросхем из совершенно разных 
семейств, и уже тем более, когда различны напряжения их питания.

Преобразование уровней можно выполнить несколькими способами:
\begin{itemize}
    \item Аналоговой схемой делителя напряжения;
    \item С использованием MOSFET;
    \item Используя микросхему.
\end{itemize}

Так как каналов, уровень которых необходимо преобразовать не много (два), выбрано преобразование напряжения с помощью полевого 
транзистора и линейного делителя напряжения. 

Так как микроконтроллер не может обеспечить достаточный ток для управления лазером (более 250 мА), то для увеличения
мощности управления лазером, используется транзистор IRZ44N. Для управления питанием выбран линейный стабилизатор
напряжения LM7805.
