\section{Разработка программного обеспечения}
\label{sec:software}

\subsection{GRBL}
Как упоминалось ранее, для управления лазерным станком используется прошивка GRBL.
Для того чтобы установить ее на микроконтроллер, необходимо выполнить следующие шаги:

\begin{enumerate}
    \item Скачать и установить Arduino IDE.
    \item Подключить плату Arduino Nano к компьютеру.
    \item Открыть Arduino IDE и выбрать плату Arduino Nano.
    \item Скачать библиотеку GRBL с официального репозитория.
    \item Установить библиотеку для Arduino IDE.
    \item Загрузить прошивку на плату с помощью файла grblUpload.ino.
\end{enumerate}

\subsection{LaserGRBL}
LaserGRBL — это бесплатное программное обеспечение с открытым исходным кодом для управления лазерными гравировальными станками,
работающими на прошивке GRBL\cite{laserGRBL}. Программное обеспечение предоставляет интерфейс для настройки скорости передачи данных, 
отображает загруженное имя файла и прогресс гравировки, может оценивать время гравировки, а также позволяет вручную вводить команды G-кода. 
С помощью управления перемещением лазера и можно обеспечить точное позиционирование при настройке устройства. 
Кнопки сброса, поиска нуля и разблокировки упрощают управление, 
а функции приостановки и возобновления позволяют обезопасить человека, пользующегося станком.  
LaserGRBL поддерживает различные форматы файлов, которые можно конвертировать 
в G-код для выполнения гравировки:

\begin{itemize}
    \item BMP;
    \item JPG;
    \item PNG;
    \item DXF;
    \item SVG.
\end{itemize}

Программу можно установить с официального сайта разработчика\cite{laserGRBL}.
Для первого запуска станка, необходимо произвести его настройку в LaserGRBL.
Вначале необходимо открыть программу и подключить микроконтроллер через порт USB Type-C к компьютеру.
Далее программа сама найдет устройство и подключится к нему. Стандартное значения скорости передачи данных 
составляет 115200 бит/с, при необходимости ее нужно вернуть к этому значению. 
После подключения станка, необходимо ввести \$\$ и нажать клавишу Enter.
В открывшемся окне необходимо ввести параметры из таблицы в Приложении \hyperref[sec:appendix:grbl]{Д}.



