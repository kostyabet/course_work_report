\subsection{Интерфейс программного средства}

Для проектирования интерфейса программного средства было принято решение использовать, как основу, библиотеку компонентов Material UI
\cite{muiDocs}. Данная библиотека предоставляет готовые компоненты, которые позволяют быстро и удобно создавать интерфейсы, а также поддерживает
адаптивный дизайн, что позволяет использовать приложение на различных устройствах.

Для проектирования архитектуры было принято решение использовать паттерн Feature-Sliced Design.

\subsubsection{Блочные компоненты}

Блочные компоненты представляют собой основные элементы интерфейса, которые используются для построения страниц приложения.
В данном программном средстве будут использовать следующие блочные компоненты:
\begin{itemize}
    \item Stack - группировка части информации;
    \item Card - карточка, которая содержит информацию;
    \item Grid - для работы с сетками;
    \item Table - вывод информации в виде таблиц;
\end{itemize}

Блочные компоненты могут содержать в себе другие компоненты, а также стили и логику, которые необходимы для их работы.

\subsubsection{Компоненты управления}

Компоненты управления представляют собой элементы интерфейса, которые позволяют пользователю взаимодействовать с приложением.
В данном программном средстве будут использовать следующие компоненты управления:
\begin{itemize}
    \item Button - кнопка, которая выполняет действие;
    \item Input - поле ввода информации;
    \item Select - выпадающий список;
\end{itemize}

Для работы с формами будет использовать стандартная библиотека react-hook-form, которая позволяет удобно и быстро работать с формами,
а для работы с валидацией будет использоваться библиотека Yup, которая позволяет удобно и быстро работать с валидацией форм.

\subsubsection{Компоненты отображения}

Компоненты отображения представляют собой элементы интерфейса, которые отображают информацию пользователю.
В данном программном средстве будут использовать следующие компоненты отображения:
\begin{itemize}
    \item Typography - для отображения текста;
    \item Avatar - для отображения аватаров пользователей;
    \item Icon - для отображения иконок;
    \item Snackbar - для отображения уведомлений пользователю;
    \item CircularProgress - для отображения индикатора загрузки;
    \item Tooltip - для отображения подсказок;
\end{itemize}