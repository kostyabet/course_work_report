\section{Разработка структурной схемы}
\label{sec:struct}

\subsection{Постановка задачи}
Для того, чтобы составить структуру разрабатываемого устройства, необходимо выделить функции, 
которые будет выполнять устройство, затем определить компоненты и связь между ними исходя из данных функций. 
Результаты можно посмотреть на структурной схеме, представленной в приложении А. 
Задачей данного проекта является реализация устройства гравировального лазерного станка и числового программного управления для него. 
Исходя из этого, были выделены следующие функции, которые должно выполнять данное устройство: 
\begin{itemize}
    \item Преобразование входного напряжения;
    \item Формирование сигнала для управления двигателями;
    \item Формирование сигнала для управления лазером.
\end{itemize}

\subsection{Определение компонентов структуры устройства}
Компоненты структуры устройства выбираются исходя из функций, определенных в постановке задачи. Проанализировав выделенные функции, были определены следующие компоненты, представленные ниже:
\begin{itemize}
    \item Модули управления двигателями;
    \item Шаговые двигатели;
    \item Микроконтроллер;
    \item Модуль управления питанием;
    \item Модуль управления лазером;
    \item Лазер.
\end{itemize}

\subsection{Взаимодействие компонентов устройства}
Задача модуля питания - предоставить качественную силовую линию\cite{schemt_1} всем потребителям на схеме. 
К ним относятся микроконтроллер, преобразователи уровней, модули управления шаговыми двигателями, лазер. 
Микроконтроллер генерирует необходимый сигнал на модули управления моторами и преобразующий транзистор, 
который в свою очередь подает необходимый ток на лазер. 