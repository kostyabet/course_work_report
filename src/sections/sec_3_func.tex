\section{Разработка функциональной схемы}
\label{sec:func}

\subsection{Блок питания}
Блок питания обеспечивает питанием все элементы устройства. Он состоит из следующих элементов:
\begin{itemize}
    \item Адаптер питания 12 В, 2 А;
    \item Конденсатор 1000 мкФ, 16 В;
    \item Линейный стабилизатор напряжения LM7805.
\end{itemize}
Стабилизатор напряжения преобразует входное напряжение и питает блок микроконтроллера выходным напряжением 5 В\cite{schemt_3}.
Драйвера шаговых двигателей потребляют 12 вольт напрямую от источника питания.

\subsection{Блок микроконтроллера}
Блок микроконтроллера содержит непосредственно платформу Arduino Nano с микроконтроллером ATmega328.
Он подключается по разъему USB Type-C к компьютеру для загрузки программного обеспечения и управления устройством.
С помощью Arduino IDE на плату загружается прошивка GRBL, которая обеспечивает управление шаговыми двигателями и лазером.
Выходной сигнал микроконтроллера подается на затвор полевого транзистора, тем самым преобразуя напряжение 5 В 
в управляющий ток для лазера.

\subsection{Блок управления лазером} 
Блок преобразования уровней предназначен для согласования уровней сигналов, работающих на различных уровнях напряжения. 
Он состоит из следующих элементов:
\begin{itemize}
    \item Резистор 47 Ом;
    \item Резистор 10 кОм;
    \item MOSFET транзистор IRZF44N.
\end{itemize}
Сигнал информционного выхода микроконтроллера проходит через резистор 47 Ом и подается на затвор транзистора.
На исток подается напряжение 12 В, а на сток подключается отрицательный выход лазера.
Транзистор включает лазер, подавая на него питание 4,7 В и 250 мА.
Таким образом реализуется схема подключения транзистора с общим истоком\cite{schemt_1}.

\subsection{Блок управления двигателями}
Данный блок состоит из драйверов шаговых двигателей и самих моторов.
Драйвера шаговых двигателей подключаются к блоку микроконтроллера и управляют напряжением на двух моторах.
Модули моторов двигают платформу вдоль оси Y и лазер по оси X.
