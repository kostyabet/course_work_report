\sectionCenteredToc{Введение}
\label{sec:intro}

Семимильными шагами ступает прогресс по планете Земля. Каждый день мы можем наблюдать,
как в мире происходит так называемая «цифровая революция», которая
началась еще в последних десятилетиях прошлого века. Связана она с распространением
информационных технологий и проникновением их во все сферы жизни общества.


В современном мире технологий брендированные товары стали неотъемлемой частью нашей повседневной жизни.
% К таким вещам относятся предметы интерьера, украшения, элементы одежды, канцелярия и многое другое. 
Во многих фирмах и компаниях сувенирная продукция является традиционной и обязательной частью 
корпоративной культуры. Для создания сувенирной продукции разрабатывается специальный дизайн, заказываются 
большие партии печатной продукции, используется печать и вышивка на одежде, а также гравировка логотипов 
на металлических и деревянных предметах.


Лазерная гравировка представляет собой метод нанесения изображения на изделие с помощью сфокусированного лазерного луча.
Линза лазера фокусирует световой луч на определенном месте и он с помощью нагревания наносит в этом месте отметку.
На рынке услуг лазерная гравировка, наряду с печатью, занимает значительное место. 
С помощью лазерной гравировки создаются раскладки на компьютерных и ноутбучных клавиатурах, 
маркируются отдельные детали и продукты, персонализируются трофеи и награды.


Данный тип гравировки обладает рядом преимуществ по сравнению с механической: благодаря автоматизации 
и недорогих материалов снижается стоимость и время нанесения изображения, отсутствие
непосредственного контакта с поверхностью материала становится облегчает гравировку неудобных и 
труднодоступных мест, лазер повышает точность рисунка, а использование числового управления уменьшает 
количество брака и погрешностей.


Но, несмотря на преимущества, лазерная гравировка имеет и недостатки: достаточно сложно контроллировать 
глубину гравировки, на изогнутой поверхности могут появляться искажения и брак из-за особенностей 
фокусного расстояния луча лазера,а также возможны небольшие деформации для материалов с низкой температурной 
стоикостью.


Данный курсовой проект поможет разобраться и погрузиться изнутри в мир современной схемотехники, 
исследовать способы нанесения изоображения лазерным лучом на различные материалы, а также попробовать
создать собственную ЧПУ систему для гравирования предметов. В рамках работы будут рассмотрены теоретические 
аспекты и проведены практические эксперименты, что позволит получить всестороннее представление о процессе 
лазерной гравировки и ее применении в различных областях.
